\documentclass{beamer}
\usetheme{metropolis}

\usepackage{amsmath, amssymb, amsthm}
\usepackage{cite, graphicx, latexcolors}
\usepackage{ragged2e, microtype}
\usepackage{url, parskip, hyperref}
\usepackage{array, tikz, flowchart}
\usepackage[dvipsnames]{xcolor}

\hypersetup{colorlinks=true,}
\setbeamercolor{title}{fg=orange}

\usetikzlibrary{arrows.meta, calc, chains, quotes, positioning, shapes.geometric}
\tikzset{FlowChart/.style={
    node distance = 5mm and 10mm,
    start chain = A going right,
    base/.style = {draw, minimum width=16mm, minimum height=19mm, align=center, on chain=A, text width=4cm},
    process/.style = {base},
    rounded/.style = {base, rounded corners},
    every edge quotes/.style = {auto=right}
}}

\title[NMPC Implementation]{
    \textbf{Implementation of Nonlinear MPC on a \\
                Quadreped robot}
}
\subtitle[Presentation]{\textcolor{brown}{
    \textbf{FoR Project Presentation - 3} \\
}}
\author[BRAMA]{%
Team-5 BRAMA \quad \scriptsize \\
\begin{tabular}{lllll}
    Biswadeep &
    Rithwik &
    Akhil &
    Moin &
    Ashrith
\end{tabular}
\vspace{2em}
}
\institute[IISc]{    
    Robert Bosch Centre for Cyber Physical Systems,\\
    Indian Institute of Science
}
\date{\scriptsize\today}
\begin{figure}
    \includegraphics[width=0.25\linewidth]{Common/rbccps.png}
\end{figure}


\begin{document}

% \section*{Titlepage}
\begin{frame}\titlepage\end{frame}\normalfont


% \section*{Outline}
\begin{frame}{Outline}
	\begin{enumerate}
            \item Review
		\item Progress
            \item Objectives
            \item Understanding
            \item Demonstration
	\end{enumerate}
\end{frame}\normalfont
%\section*{Progress}
\begin{frame}{Progress}
        \begin{enumerate}
            \item ROS
        \end{enumerate}
\end{frame}\normalfont
%\section*{Objectives}
\begin{frame}{Objectives}
        \begin{enumerate}
           % \item{Changed the reference paper from the earlier 2019 paper by the same author.}
            \item{Attempting to get the quadruped to stand upright.}
            \item{Improve upon the existing codebase to enable walking motion}
            \item{Implement a QP formulation}
            % \item{Implement a swing controller based foot placement strategy}
            % \item{Porting the code for a \texttt{Unitree Go1} robot}
        \end{enumerate}
\end{frame}\normalfont

\begin{frame}{MPC overview}
\begin{figure}
    \centering
    \includegraphics[width=\linewidth]{Presentation-2/rf-mpc-flow.png}
    \caption{Overview of the MPC control system}
\end{figure}
\end{frame}

\begin{frame}{RBD model}
\begin{figure}
    \centering
    \includegraphics[width=0.6\linewidth]{../Presentation-1/Illustration-RBD-model.png}
    \caption{Illustration of the 3D Single RBD model}
    \label{fig:rbd-model}
\end{figure}
\begin{equation*}
    \mathbf{\dot{x}} = f(\mathbf{x}, \mathbf{u}) = 
    \begin{bmatrix}
    \mathbf{\dot{p}} \\
    \mathbf{\ddot{p}} \\
    \mathbf{\dot{R}} \\
    \boldsymbol{\dot{\omega}}
    \end{bmatrix}
    =
    \begin{bmatrix}
    \mathbf{\dot{p}} \\
    \frac{1}{M} \mathbf{f} - \mathbf{a_g} \\
    \mathbf{R} \cdot \hat{\boldsymbol{\omega}} \\
    {}^{B} \mathbf{I}^{-1}(R^T \boldsymbol{\tau} - \hat{\boldsymbol{\omega}} {}^{B} \mathbf{I} \boldsymbol{\omega})
    \end{bmatrix}
\end{equation*}
\begin{itemize}
    \item Gazebo
\end{itemize}
\end{frame}


\begin{frame}{Non-Linear MPC formulation}
\begin{equation*}
\begin{aligned}
    & \text{minimize} \quad \sum_{k=1}^{N-1} \ell(\mathbf{x_k}, \mathbf{u_k}) + \ell_T(\mathbf{x_N}) \\
    & \text{subject to} \quad \mathbf{x_{k+1}} = g(\mathbf{x_k}, \mathbf{u_k}), \quad \mathbf{x_1} = \mathbf{x_\text{op}}, \\
    & \mathbf{x_k} \in \mathbb{X}, \quad k = 1, 2, \dots, N, \\
    & \mathbf{u_k} \in \mathbb{U}, \quad k = 1, 2, \dots, N-1
\end{aligned}
\end{equation*}
\begin{itemize}
    \item Stage cost: $ \ell(\mathbf{x_k}, \mathbf{u_k}) = \lVert \mathbf{x_k} - \mathbf{x_{d, k}} \rVert ^ 2 \mathbf{Q_x} + \lVert \mathbf{u_k} - \mathbf{u_{d, k}} \rVert ^ 2 \mathbf{R_u}$
    \item Terminal cost: $ \ell_T $
    \item Error terms $\leftrightarrow$ Quadratic forms
\end{itemize}
\end{frame}


\begin{frame}{Non-Linear MPC formulation}
\begin{itemize}
    \item Using \texttt{qpsolvers} library, \texttt{quadprog} solver.
    \item 
\begin{figure}
    \centering
    \includegraphics[width=0.75\linewidth]{Presentation-2/WhatsApp Image 2024-09-30 at 17.18.57_d0017338.jpg}
  
\end{figure}
\end{itemize}
\end{frame}


\begin{frame}{Non-Linear MPC formulation}
\begin{block}{Contraints}
\begin{itemize}
    \item Normal forces should be non-negative when foot is in contact
    \item Tangent forces within the friction cones.\\
        (Linearised version uses friction pyramids).
\end{itemize}
\end{block}
\end{frame}

\begin{frame}{Variation-based Linearisation}
\begin{block}{Assumption}
    Predicted variables are close to the operating point.
\end{block}
\begin{align}
\mathbf{R_k} & \approx \mathbf{R_{op}} (\mathbf{I} + \delta \mathbf{R_k})
\\ \implies
\dot{\mathbf{R_k}} & \approx \mathbf{R_{op}}\mathbf{\hat \omega_{op}} + \mathbf{R_{op}}\mathbf{\hat\omega_{op}}\mathbf{R_k} + \mathbf{R_{op}}\widehat{\delta\mathbf{\omega_k}}
\end{align}
\begin{equation}
    ^B I \dot{\omega}_k = R_{op}^T \tau_{op} + \delta R_k^T \tau_{op} + R_{op}^T \delta \tau_k - \hat{\omega}_{op} {}^B I \omega_{op} - \delta \hat{\omega}_k {}^B I \omega_{op} - \hat{\omega}_{op} {}^B I \delta \omega_k
\end{equation}
\end{frame}

\begin{frame}{Vectorisation}
\begin{itemize}
    \item Converting matrix-matrix products to matrix-vector products, so that it is easier for the the QP solver.
    \item Lemmas:
        \begin{align}
            vec (\mathbf{AB}) & = (\mathbf{I} \otimes \mathbf{A}) vec(\mathbf{B})
            \\ 
            vec (\mathbf{AXB}) & = (\mathbf{B}^T \otimes \mathbf{A}) vec(\mathbf{X})
        \end{align}
    \item Implications:
    We can compute the terms in the expressions by defining a few constant matrices
\end{itemize}
\end{frame}

\begin{frame}{Simulations}
\begin{itemize}
    \item Plan to do it, similar to the paper
    \begin{itemize}
        \item Walking trot
        \item Bounding
    \end{itemize}
\end{itemize}
\end{frame}

\begin{frame}
	\LARGE{Thank You!}
\end{frame}\normalfont

\end{document}
